% !TEX TS-program = lualatex
\documentclass{article}

\usepackage[
	font={Palemonas MUFI},
	mufifour
]{mufi}

\setlength{\parindent}{0pt}

\newcommand\display[1]{%
	\texttt{\ignorespaces\detokenize{#1}}: [#1]%
}

\begin{document}

\section{Normal Unicode characters}
\display{Ā} --- 
\display{B} --- 
\display{Ċ} --- 
\display{ā} --- 
\display{b} --- 
\display{ċ} --- 

\section{MUFI 3.0 characters}
\display{\msignflour{}} --- 
\display{\mufi{msignflour}} --- 
\display{\mufi{F2F3}}\\
\display{\yrgmainstrok{}} --- 
\display{\mufi{yrgmainstrok}} --- 
\display{\mufi{F233}}


\section{MUFI 4.0 characters}
Note that the currently installed Palemonas MUFI font on this machine does not have the OEligogon character.

\vspace{12pt}
\display{\OEligogon{}} --- 
\display{\mufi{OEligogon}} --- 
\display{\mufi{E262}}\\
\display{\UUlig{}} --- 
\display{\mufi{UUlig}} --- 
\display{\mufi{E8C6}}\\
\display{\helbing{}} --- 
\display{\mufi{helbing}} --- 
\display{\mufi{F2FB}}


\section{Starred vs. unstarred}
The unstarred commands use the configured font, whereas the starred version simply uses the current document font. Note the difference in the \@ signs as well as the missing \texttt{msignflour} in the document font.

\vspace{12pt}
\display{\mufi{0040}} --- 
\display{\mufi*{0040}}\\
\display{\mufi{0041}} --- 
\display{\mufi*{0041}}\\
\display{\msignflour{}} --- 
\display{\msignflour*{}}\\

\end{document}
