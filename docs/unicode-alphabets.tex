% !TEX TS-program = lualatex
\documentclass{article}

\usepackage{biblatex}
\addbibresource{unicode-alphabets.bib}

\usepackage{showexpl}
\usepackage{float}
\usepackage{moreverb}

\title{Unicode Alphabets}
\author{Mikkel Eide Eriksen}

\usepackage{fontspec}
\newfontfamily\mufifont{Palemonas MUFI}
\newfontfamily\titusfont{TITUS Cyberbit Basic}

\usepackage[
	CYFI,
	MUFI,
	TITUS,
%	disable entity macros,
	fonts={
		CYFI=\titusfont,
		MUFI=\mufifont,
		TITUS=\titusfont,
	},
]{unicode-alphabets}

\begin{document}

\maketitle

\section{Preface}

While Unicode supports the vast majority of use cases, there are certain specialized niches which require characters and glyphs not (yet) represented in the standard.

Thus the Private Use Area (PUA) at code points E000--F8FF, which enables third parties to define arbitrary character sets.

This package allows configuring a number of macros to enter characters from the PUA by name or code point.

\section{Setup}

The package is configured in the following manner:

\begin{verbatim}
\usepackage[⟨options⟩]{unicode-alphabets}
\end{verbatim}

Where \verb|options| must be one or more of the following character sets, some of which may be mutually (in-)compatible. See references for further detail on each.

\begin{description}

% \item[CSUR] - https://www.evertype.com/standards/csur/
\item[CYFI] Early Cyrillic glyphs.
% DANIA ???
% \item[LINCUA] Includes the mutually compatible (?) character sets CYFI, DANIA, MUFI, and TITUS
\item[MUFI] The Medieval Unicode Font Initiative.
% \item[SIL] SIL International.
\item[TITUS] Thesaurus Indogermanischer Text- und Sprachmaterialien.
% \item[UNZ] Normung von Sonderzeichen.

\end{description}

There is no default, since future versions of this package may supply more character sets that are incompatible with the above.

\clearpage
Additionally, one may configure different fonts for each character set, as in the following example:

\begin{figure}[H]
\centering
\begin{verbatimtab}
\usepackage[
	MUFI,
	TITUS,
	fonts={
		MUFI=\mufifont,
		TITUS=\titusfont,
	},
]{unicode-alphabets}
\end{verbatimtab}
\caption{Example setup}
\end{figure}

If no fonts are configured, the document font will be used.

Finally, the \verb|MUFI| character set has defined entity names, which result in the creation of macros for each character (see \verb|msignfour| in the following example). These can be suppressed with the \verb|disableentitymacros| option.

\section{Usage}

Each set defines a macro in the following manner. Let's use \verb|MUFI| as an example.

By default, a macro with the lower-case name of the character set is defined: \verb|\mufi{}|. It can then be used to display characters from the given set:

\begin{figure}[H]
\centering
\begin{LTXexample}[varwidth=true]
\mufi{FLOURISHED SMALL LETTER M SIGN}\\
\msignflour{}
\mufi{msignflour}
\mufi{F2F3}
\end{LTXexample}
\caption{Example usage}
\end{figure}

Additionally, starred versions of each macro are defined, which suppress using the configured font, falling back to the document font.

\nocite{*}
\printbibliography

\end{document}
