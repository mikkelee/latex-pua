% !TEX TS-program = lualatex
\documentclass{article}

\usepackage{fontspec}
\newfontfamily\mufifont{Palemonas MUFI}
\newfontfamily\titusfont{TITUS Cyberbit Basic}

\usepackage[
	CYFI,
	MUFI,
	fonts={
		CYFI=\titusfont,
		MUFI=\mufifont,
	},
]{unicode-alphabets}

\setlength{\parindent}{0pt}

\newcommand\display[1]{%
	\texttt{\ignorespaces\detokenize{#1}}: \fbox{#1}%
}

\begin{document}

\section{Normal Unicode characters}
\display{Ā} --- 
\display{B} --- 
\display{Ċ} --- 
\display{ā} --- 
\display{b} --- 
\display{ċ} --- 

\section{MUFI 3.0 characters}
\display{\mufi{FLOURISHED SMALL LETTER M SIGN}}\\
\display{\msignflour{}} --- 
\display{\mufi{msignflour}} --- 
\display{\mufi{F2F3}}

\vspace{8pt}
\display{\mufi{LATIN SMALL LETTER Y WITH RIGHT MAIN STROKE}}\\
\display{\yrgmainstrok{}} --- 
\display{\mufi{yrgmainstrok}} --- 
\display{\mufi{F233}}

\section{MUFI 4.0 characters}
Note that the currently installed Palemonas MUFI font on this machine does not have the OEligogon character.

\vspace{12pt}
\display{\mufi{LATIN CAPITAL LIGATURE OE WITH OGONEK}}\\ 
\display{\OEligogon{}} --- 
\display{\mufi{OEligogon}} --- 
\display{\mufi{E262}}

\vspace{8pt}
\display{\mufi{LATIN CAPITAL LIGATURE UU}}\\
\display{\UUlig{}} --- 
\display{\mufi{UUlig}} --- 
\display{\mufi{E8C6}}

\vspace{8pt}
\display{\mufi{HELBING SIGN}}\\
\display{\helbing{}} --- 
\display{\mufi{helbing}} --- 
\display{\mufi{F2FB}}


\section{Starred vs. unstarred}
The unstarred commands use the configured font, whereas the starred version simply uses the current document font. Note the difference in the \@ signs as well as the missing \texttt{msignflour} in the document font.

\vspace{12pt}
\display{\mufi{0040}} --- 
\display{\mufi*{0040}}

\display{\mufi{0041}} --- 
\display{\mufi*{0041}}

\display{\msignflour{}} --- 
\display{\msignflour*{}}

\section{CYFI characters}
\display{\cyfi{CYRILLIC CAPITAL LETTER REVERSED A}}\\
\display{\cyfi{F330}}


\end{document}
